\documentclass[11pt,a4paper]{article} %title page
\usepackage[margin = 2cm]{geometry}


\usepackage{titletoc}
\usepackage[pagestyles]{titlesec}
\usepackage[no-math]{fontspec} 

%加這個就可以設定字體
\usepackage[CJKnumber, CJKchecksingle]{xeCJK} 
\usepackage[usenames,dvipsnames,svgnames,table]{xcolor}

\usepackage{paralist} % enhanced list environment
\usepackage{textcomp} % for including symbols
\usepackage{multicol} 
\usepackage{multirow}
\usepackage{threeparttable} %add footnote under tables
\usepackage{booktabs, tabularx} %extend control of width of columns 
%\usepackage{warpcol}%for decimal points alignment in tables


% packages about math
\usepackage{amssymb, amsmath, mathcomp} 

%for figs
\usepackage{placeins} % to flush all the floats before
\usepackage{graphicx}
\usepackage{wrapfig} %控制圖繞文 
\usepackage{float} % controlling the place of figure
\restylefloat{figure}

% for caption
\usepackage{caption, subcaption}
\usepackage{hyperref}

% for Reference
\usepackage[square, numbers, sort&compress]{natbib}
\usepackage[page,titletoc]{appendix} %toc for Appendices


%讓中英文字體分開設置
%設定中文的字型,可以直接輸入系統裡有的字型
\setCJKmainfont[BoldFont=DFHei Std W7]{DFHei Std W5}
\newCJKfontfamily{\kai}[BoldFont=DFKaiShu Std W7]{DFBiaoKaiShu Std W5}
\setmonofont[Mapping=tex-text]{Inconsolata}
%\setromanfont[Mapping=tex-text,BoldFont=Adobe Caslon Pro Bold]{Adobe Caslon Pro}
\setromanfont[Mapping=tex-text,BoldFont=Adobe Garamond Pro Bold]{Adobe Garamond Pro}
\setsansfont[Mapping=tex-text]{FuturaStd-Medium}
\newfontfamily{\optima}{OptimaLTStd}

\XeTeXlinebreaklocale "zh"
\XeTeXlinebreakskip = 0pt plus 1pt
%上面這二行,中文才能自動換行

\newcommand{\horrule}[1]{\rule{\linewidth}{#1}} 	% Horizontal rule
\newcommand{\nonumsubsec}[1]{\phantomsection \addcontentsline{toc}{subsection}{#1} \subsection*{\optima #1}}

\renewcommand{\thesection}{\Roman{section}}
\titleformat{\section}{\Large\bfseries}{\thesection.}{0.5em}{}
\renewcommand\figurename{圖}
\renewcommand\tablename{表}
%\renewcommand\lstlistlistingname{List of Source Codes}
%\renewcommand\lstlistingname{Code}


\newpagestyle{main}{\sethead{\small\kai\optima\sectiontitle}{\optima 2013 ML Final Project}{} \setfoot{}{\optima\thepage}{}\headrule}
%\newpagestyle{main}{\sethead{}{}{} \setfoot{}{\thepage}{}}

\renewcommand{\today}{\number\year 年 \number\month 月 \number\day 日}

%% Short-keys for easy typing
\newcommand{\figref}[1]{圖~\ref{#1}}
\newcommand{\tabref}[1]{表~\ref{#1}}

\newcommand{\scfi}[2] {\ensuremath{#1 \times 10^{#2}}}
\renewcommand{\t}[1]{\text{#1}}
\newcommand{\tk}{\text{k}}\newcommand{\tm}{\text{m}}
\newcommand{\tn}{\text{n}}\newcommand{\tM}{\text{M}}

\newcommand{\insertfig}[4]{\begin{figure}[H]
	\begin{center}
	\includegraphics[width = #2\textwidth]{#1}
	\caption{#3}\label{#4}
	\end{center}\end{figure}}
\newcommand{\insertfigx}[5]{\begin{figure}[#5]
	\begin{center}
	\includegraphics[width = #2\textwidth]{#1}
	\caption{#3}\label{#4}
	\end{center}\end{figure}}

\newcommand{\inserttable}[1]{
	\begin{table}[H]\vspace{-5pt}\begin{center}
	\input{#1}\vspace{-10pt}\end{center}\end{table}\vspace{-10pt}}
\newcommand{\inserttablex}[3]{
	\begin{table}[H]\begin{center}\caption{#2}\vspace{-10pt}\label{#3}
	\input{#1}\end{center}\vspace{-10pt}\end{table}}

%% end short-keys
\newcommand{\link}[2]{ \href{#1}{\textcolor{Gray}{#2}}}
\newcommand{\statatab}[1]{
\begin{table}[H]\centering\begin{threeparttable}
\input{#1}\end{threeparttable}\end{table}}

\usepackage{mathabx}
%\pagestyle{main}
\begin{document}
%\setlength{\parindent}{0cm}
\setlength{\baselineskip}{1.5em}
\setlength{\parskip}{0.5em}

\begin{center}
{\optima\LARGE Machine Learning~HW1}\\[0.5em]
電機五 王亮博 B98901114
\end{center}
%%%%%% 正文 %%%%%%
\subsection*{p1. [a]}
\begin{enumerate}[(i)]
\item 可以用明確簡單的演算法來判斷是否為質數。例如檢查比 $\sqrt{x}$ 小的所有整數能否整除 $x$。
\item[(iii)] 存在明確的物理法則來決定一個物體落下的時間 $t = \sqrt{2h/g}$
\end{enumerate}
其餘的情況都很適合使用 Machine Learning。

\subsection*{p2. [a]}
把使用者過去的評價視為 label,根據這些資訊對尚未評價的書籍去作 labeling,並推薦那些預測評價較好的書籍。故是 supervised learning。

\subsection*{p3. [c]}
我們事先並不知道持卡者哪些購買行為是「尋常」與「不尋常」,即我們沒有這樣 label 的資訊,故為一種 unsupervised learning。

\subsection*{p4. [b]}
模仿自然的學習,例如我們可以給機器一個節奏與曲調,並請機器產生音高的時間序列。這樣的時間序列 $y_n$ 不一定能稱為「音樂」(不符合樂理,樂句很奇怪),人可以給予獎勵說哪些序列比較好聽($\tilde{y_n}$)。此為一種 reinforcement learning。

\subsection*{p5. [d]}
一個帳戶持有者的借貸上限,基本上決定於一個人的償債能力,可以說大致上與其現帳戶存款成正比。存款越多,貸款上限就越高。銀行可以使用一個簡單的公式來決定借貸上限(例如:$\text{上限} = \text{存款} \times 60\%$)。故本問題不必使用到機器學習。

\subsection*{p6. [d]}
\[
E_{OTS}(g, f) = \frac1L\sum^L_{l=1} \left\ldbrack g(\mathbf{x}_{N+l}) \neq f(\mathbf{x}_{N+l}) \right\rdbrack 
    = \frac1L\sum^L_{l=1} \left\ldbrack g(\mathbf{x}_{N+l}) \neq 1 \right\rdbrack,
\]
which 
\[  
\sum^L_{l=1} \left\ldbrack g(\mathbf{x}_{N+l}) \neq 1 \right\rdbrack = \text{\# evens between } (N, N+L].
\]

We claim that \# evens between $(N, N+L]$ equals to $\left\lfloor \frac{N+L}2\right\rfloor - \left\lfloor \frac{N}2\right\rfloor$. It can be shown by
given $m, n\in \mathbb{N}, m > n$, parity of $N + L$ and $N$ falls into the following 4 cases:

\begin{table}[H]
\centering
\begin{threeparttable}
	\begin{tabular}{cccc}
		\toprule
		$N+L$ 	& $N$  	& \# evens in $(N, N+L]$	& $\left\lfloor \frac{N+L}2\right\rfloor - \left\lfloor \frac{N}2\right\rfloor$ \\
		\midrule
		$2m$ & $2n$ & $m-n$ & $m-n$\\
		$2m+1$ & $2n$ & $m-n$ & $m-n$\\
		$2m$ & $2n+1$ & $m-n$ & $m-n$\\
		$2m+1$ & $2n+1$ & $m-n$ & $m-n$\\
		\bottomrule
	\end{tabular}
\end{threeparttable}
\end{table}

Thus
\[
E_{OTS}(g, f) = \frac1L \left(\left\lfloor \frac{N+L}2\right\rfloor - \left\lfloor \frac{N}2\right\rfloor\right).
\]


\subsection*{p7. [c]}
First we should show that given $\mathbf{x}_n = x_1, x_2, \dotsc, x_n$, the maximum number of different $f(\mathbf{x}_n)$s is $2^n$ by mathematical induction. Starting with $n = 1$, there are two $f$s,

\begin{table}[H]
\centering
\begin{tabular}{c|c|c}
        & $f_1$ & $f_2$\\
\hline
$x_1$   & $\circ$ & $\times$ 
\end{tabular}
\end{table}

Assume that at $n=k$ the statement holds true, so there are maximum $2^k$ possible $f$s. Given $n=k+1$, we can construct distinct $2^{k}$ $f$s with $f(x_{k+1}) = \circ$ and another $2^{k}$ $f$s with $f(x_{k+1}) = \times$, which is guaranteed by the case $n = k$. Thus we can construct maximally $2^{k+1}$ different $f$s and we proof the statement.

Now we want to construct all possible $f\colon\mathcal{X}\to\mathcal{Y}$ such that $f(\mathbf{x}_n) = y_n, \forall (x_n, y_n) \in \mathcal{D}$. It means outputs of $f(x_1), \dotsc, f(x_N)$ have been fixed. So now we want to construct as many different output of $f(x_{N+1}), \dotsc, f(x_{N+L})$ as possible. By the previous proof we can construct maximally $2^L$ different $f$s.

\subsection*{p8. [a], [d]}
We will prove  $E_{OTS}(g, f) = \frac{k}{L}$ is ${L \choose k}$ by mathematical induction. Given $L$ for case $k=0$, $\overbrace{\circ\circ\circ\dotsm\circ}^{\text{total length} L}$ with total 0 $\times$s has only $1 = {L \choose 0}$ possibility. Assume the statement holds true for case $k = m$. For $k = m+1$, each $\overbrace{\circ\times\circ\dotsm\times\circ}^{\text{length } L, m+1 \times\text{s}}$ falls into either two cases:
\begin{align*}
\circ&\overbrace{\circ\times\circ\circ\dotsm\times\circ}^{\text{length } L-1, m+1 \times\text{s}} \\
\times&\overbrace{\circ\times\circ\circ\dotsm\times\circ}^{\text{length } L-1, m \times\text{s}}
\end{align*}
Each case has ${L-1 \choose m+1}$ and ${L-1 \choose m}$ different $f$s respectively. So the total number of $f$s is
\[
{L-1 \choose m+1} + {L-1 \choose m} = {L \choose m+1}.
\]
Hence by mathematical induction the statement holds true for $0 \leq k \leq L$.

For [c] and [d] we will first prove
\[
	\sum^{L}_{k=0} {L \choose k} = 2^L.
\]
Case $L=1$,
\[
    \sum^1_{k=0} {1 \choose k} = 1 + 1 = 2 = 2^1.
\]
Assume the statement holds true for $L = l$. For case $L = l+1$,
\begin{align*}
\sum^{l+1}_{k=0}{l+1 \choose k} &= \sum^{l}_{k=1}{l+1 \choose k} + {l+1 \choose l+1} + {l+1 \choose 0}\\
    &= 2 +  \sum^{l}_{k=1}{l\choose k} + \sum^{l}_{k=1}{l\choose k-1} \\
    &= 2 + \left(2^l - {l\choose 0}\right) + \sum^{l-1}_{j=0}{l\choose j} \\
    &= 2 + \left(2^l - 1\right) + \sum^{l}_{j=0}{l\choose j} - {l\choose l} \\
    &=2 + 2^l -1 + 2^l -1 \\
    &= 2^{l+1}
\end{align*}
Thus it is proved by mathematical induction.

Then we are going to prove that
\[
    \sum^{L}_{k=1}\frac{k}{L}{L\choose k} = 2^{L-1}.
\]
For $L = 2$,
\[
    \sum^{2}_{k=1}\frac{k}{2}{2\choose k} = \frac12{2\choose 1} + \frac22{2\choose 2} = 2 = 2^{2-1}.
\]
Assume the statement holds true for $L = l$. For case $L = l+1$,
\begin{align*}
\sum^{l+1}_{k=1}\frac{k}{l+1}{l+1\choose k}
&=\sum^{l}_{k=1}\frac{k}{l+1}{l+1\choose k} + \frac{l+1}{l+1}{l+1\choose l+1}\\
&=\sum^{l}_{k=1}\left[\frac{k}{l+1}\left({l\choose k} + {l\choose k-1}\right)\right] + 1\\
&=\frac{l}{l+1}\sum^{l}_{k=1}\frac{k}{l} {l\choose k} + \sum^{l}_{k=1}\frac{k}{l+1} {l\choose k-1} + 1\\
&=\frac{l}{l+1}2^{l-1} + 1 
    + \sum^{l}_{k=1}\frac{1}{l+1}{l\choose k-1} 
    + \sum^{l}_{k=1}\frac{k-1}{l+1}{l\choose k-1}\\
&=\frac{l}{l+1}2^{l-1} + 1 
    + \frac1{l+1}\sum^{l-1}_{j=0}{l\choose j} 
    + \frac{l}{l+1}\sum^{l-1}_{j=0}\frac{j}{l}{l\choose j}\\
&=\frac{l}{l+1}2^{l-1} + 1
    + \frac1{l+1}\left(2^l - 1\right)
    + \frac{l}{l+1}\left(2^{l-1} - 1\right)
= 2^l
\end{align*}
Thus the statement is proved by mathematical induction. Finally the expected OTS can be computed as
\[
\mathbb{E}_f\{E_{OTS}(g, f)\} 
    = \dfrac{\sum^{L}_{k=1}\frac{k}{L}{L\choose k}}{2^L} = \dfrac{2^{L-1}}{2^L} 
    = \dfrac12.
\]

\subsection*{p9. [a]}
Let $Y = 10\nu$, then $Y$ is a random variable following $Binomial(10, \mu)$. Given $\mu = 0.5$,
\[
    \mathbb{P}(Y=5) = {10\choose5}\mu^5(1-\mu)^5 = \frac{42\times6}{2^{10}} \approx 0.24.
\]

\subsection*{p10. [b]}
Now $Y \sim Binomial(10, 0.9)$, so
\[
    \mathbb{P}(Y=9) = {10\choose9}\cdot0.9^9\cdot0.1^1 \approx 0.39.
\]

\subsection*{p11. [d]}
\begin{align*}
\mathbb{P}(\nu \leq 0.1) &= \mathbb{P}(10\nu \leq 1)\\
	&= \mathbb{P}(Y\leq1)\\
    &= \mathbb{P}(Y = 0 \cup Y=1)\\
    &= {10\choose0}\cdot0.9^0\cdot0.1^{10} + {10\choose1}\cdot0.9^1\cdot0.1^9 \\
    &\approx 9.1 \times 10^{-9}
\end{align*}

\subsection*{p12. [b]}
$-\nu \geq -0.1$, $\mu - \nu \geq 0.8$. By Hoeffding's Inequality,
\[
    \mathbb{P}(|\mu - \nu| \geq 0.8) \leq 2\exp(-2\cdot0.8^2\cdot10) = 5.52 \times 10^{-6}
\]

\subsection*{p13. [b]}
Only dice B or C has orange 1. So
\begin{align*}
\mathbb{P}(\text{5 orange 1s}) &= \mathbb{P}(\text{5 B or C})\\
    &= \mathbb{P}(\text{5 B}) + \mathbb{P}(\text{5 C}) 
        + \mathbb{P}(\text{4 B 1 C}) + \mathbb{P}(\text{3 B 2 C}) + \mathbb{P}(\text{2 B 3 C}) + \mathbb{P}(\text{1 B 4 C})\\
        &= \dfrac{1 + 1 + {5 \choose 1} + {5 \choose 2} + {5 \choose 3} + {5 \choose 4}}{4^5}\\
        &= \frac{32}{1024} = \frac8{256}
\end{align*}

\subsection*{p14. [c]}
There are some cases that no number can have same color:
\begin{enumerate}[(i)]
    \item both A and B appears
    \item both C and D appears
\end{enumerate}
Thus the probability can be rewritten as
\begin{align*}
\mathbb{P}(\text{some number is purely orange})
&= \mathbb{P}(\text{only one type of dice appears})\\ 
&\quad+ \mathbb{P}(\text{only A and C})\\
&\quad+ \mathbb{P}(\text{only A and D})\\
&\quad+ \mathbb{P}(\text{only B and C})\\
&\quad+ \mathbb{P}(\text{only B and D})\\
\end{align*}
From the previous problem, we know that
\[
\mathbb{P}(\text{only one type of dice appears}) = 4 \times \frac1{4^5}
\]
\[
\mathbb{P}(\text{only A and C}) = \mathbb{P}(\text{only A and D}) = \mathbb{P}(\text{only B and C}) = \mathbb{P}(\text{only B and D}) = \frac{30}{4^5}
\]
So $\mathbb{P}(\text{some number is purely orange}) = \dfrac{4 \times 1 + 4 \times 30}{1024} = \dfrac{31}{256}$

\subsection*{Bonus 21}
Since $y_{n(t)} \neq \text{sign}(\mathbf{w}^T_t\mathbf{x}_{n(t)})$, $y_{n(t)}\mathbf{w}^T_t\mathbf{x}_{n(t)} < 0$. We define $A = - y_{n(t)}\mathbf{w}^T_t\mathbf{x}_{n(t)}$, $A > 0$.

\begin{align*}
y_{n(t)}\mathbf{w}_{t+1}^T\mathbf{x}_{n(t)}
&= y_{n(t)}\left(
    \mathbf{w}_t + y_{n(t)}\mathbf{x}_{n(t)}\cdot\left\lfloor
        \frac{A}{\|\mathbf{x}_{n(t)}\|^2} + 1
    \right\rfloor\right)^T\mathbf{x}_{n(t)}\\
&= \overbrace{y_{n(t)}\mathbf{w}_t^T\mathbf{x}_{n(t)}}^{-A} + 
    [y_{n(t)}]^2\cdot\mathbf{x}_{n(t)}^T\left\lfloor
        \frac{A}{\|\mathbf{x}_{n(t)}\|^2} + 1
    \right\rfloor\mathbf{x}_{n(t)}\\
&= -A + [y_{n(t)}]^2\|\mathbf{x}_{n(t)}\|^2\left\lfloor
        \frac{A}{\|\mathbf{x}_{n(t)}\|^2} + 1
    \right\rfloor\\
    &=  -A + \overbrace{[y_{n(t)}]^2}^{1}\left\lfloor A + \|\mathbf{x}_{n(t)}\|^2 \right\rfloor\\
\end{align*}

Since for all $x$ we have the inequality $ x-1 < \lfloor x \rfloor \leq x$,
\[
y_{n(t)}\mathbf{w}_{t+1}^T\mathbf{x}_{n(t)}
= -A + \left\lfloor A + \|\mathbf{x}_{n(t)}\|^2 \right\rfloor
\geq -A + A + \|\mathbf{x}_{n(t)}\|^2 -1 = \|\mathbf{x}_{n(t)}\|^2 -1.
\]

Since we have defined $x_0 = 1$. Therefore, 
\begin{align*}
\|\mathbf{x}_{n(t)}\|^2
&= \|(x_{0(t)}, x_{1(t)}, \dotsc, x_{n(t)})\|^2\\
&= \sum_{i=0}^n x_{i(t)}^2 \geq x_{0(t)}^2 = 1
\end{align*}

Finally we have
\[
y_{n(t)}\mathbf{w}_{t+1}^T\mathbf{x}_{n(t)}
\geq \|\mathbf{x}_{n(t)}\|^2 -1
\geq 0
\]
\subsection*{Bonus 22}
If the data set $\mathcal{D}$ is linear separable, there exists a perfect weight $\mathbf{w}_f$ such that
\[
    y_{n(t)} = \text{sign}(\mathbf{w}_f^T\mathbf{x}_{n(t)}) \text{ and }
    y_{n(t)}\mathbf{w}_f^T\mathbf{x}_{n(t)} > 0, \forall (\mathbf{x}_{n(t)}, y_{n(t)}) \in \mathcal{D}.
\]
For simplicity, we define $A(t) = -y_{n(t)}\mathbf{w}_f^T\mathbf{x}_{n(t)}, A(t) > 0$. Then we are going to show that after an update (on finding a misclassification), we always have
\[
    \|\mathbf{w}_{t+1}\|^2 < \|\mathbf{w}_t\|^2  + \|\mathbf{x}_{n(t)}\| ^2.
\]

The proof is straight forward,
\begin{align*}
\|\mathbf{w}_{t+1}\|^2 
&= \left\|
    \mathbf{w}_t + y_{n(t)}\mathbf{x}_{n(t)}\left(\left\lfloor
        \frac{A(t)}{\|\mathbf{x}_{n(t)}\|^2} + 1
    \right\rfloor\right)^2
\right\|^2\\
&= \|\mathbf{w}_t\|^2 
    + 2\overbrace{y_{n(t)}\mathbf{w}_t^T\mathbf{x}_{n(t)}}^{< 0}\overbrace{\left\lfloor\ldots\right\rfloor}^{> 0}
    + \overbrace{y_{n(t)}^2}^{=1}\|\mathbf{x}_{n(t)}\|^2\left(\left\lfloor\ldots\right\rfloor\right)^2\\
&< \|\mathbf{w}_t\|^2
    + \|\mathbf{x}_{n(t)}\|^2\left(\left\lfloor
        \frac{A(t)}{\|\mathbf{x}_{n(t)}\|^2} + 1
    \right\rfloor\right)^2\\
&= \|\mathbf{w}_t\|^2
    + \left(\left\lfloor
        \frac{A(t)}{\|\mathbf{x}_{n(t)}\|} + \|\mathbf{x}_{n(t)}\|
    \right\rfloor\right)^2
\end{align*}
plus we always have the inequality $x-1 < \lfloor x \rfloor \leq x$,
\begin{align*}
\|\mathbf{w}_{t+1}\|^2 
&< \|\mathbf{w}_t\|^2 + \left(\left\lfloor
    \frac{A(t)}{\|\mathbf{x}_{n(t)}\|} + \|\mathbf{x}_{n(t)}\|
\right\rfloor\right)^2\\
&\leq \|\mathbf{w}_t\|^2 + \left(
    \frac{A(t)}{\|\mathbf{x}_{n(t)}\|} + \|\mathbf{x}_{n(t)}\|
\right)^2\\
&= \|\mathbf{w}_t\|^2 + \frac{A(t)^2}{\|\mathbf{x}_{n(t)}\|^2} + \|\mathbf{x}_{n(t)}\|^2 + 2A(t)\\
&< \|\mathbf{w}_t\|^2 + \|\mathbf{x}_{n(t)}\|^2
\end{align*}

If we start this modified PLA implementation from $\mathbf{w}_0 = 0$, after T updates,
\begin{align*}
\|\mathbf{w}_T\|^2 &< \|\mathbf{w}_{T-1}\|^2 + \|\mathbf{x}_{n(T-1)}\|^2\\
\|\mathbf{w}_{T-1}\|^2 &< \|\mathbf{w}_{T-2}\|^2 + \|\mathbf{x}_{n(T-2)}\|^2\\
&\ldots\\
\|\mathbf{w}_{1}\|^2 &< \|\mathbf{w}_0\|^2 + \|\mathbf{x}_{n(1)}\|^2\\
\Rightarrow \|\mathbf{w}_T\|^2 &< \sum^{T-1}_{t=0}\|\mathbf{x}_{n(t)}\|^2
    \leq T \cdot \max_{t\in T}\|\mathbf{x}_{n(0)}\|^2
\end{align*}

So we have $\|\mathbf{w}_T\| < \sqrt{T} \cdot \max\limits_{t\in T}\|\mathbf{x}_{n(0)}\|$. Also, combining all update processes,
\begin{align*}
\mathbf{w}_f^T\mathbf{w}_T
&= \mathbf{w}_f^T\mathbf{w}_{T-1} + y_{n(T-1)}\mathbf{w}_f^T\mathbf{x}_{n(T-1)}\left\lfloor
    \frac{A(T-1)}{\|\mathbf{x}_{n(T-1)}\|^2} + 1
\right\rfloor = \ldots \\
&= \overbrace{\mathbf{w}_f^T\mathbf{w}_0}^{0} + \sum^{T-1}_{t=0} y_{n(t)}\mathbf{w}_f^T\mathbf{x}_{n(t)} \left\lfloor
    \frac{A(t)}{\|\mathbf{x}_{n(t)}\|^2} + 1
\right\rfloor\\
&= \sum^{T-1}_{t=0} y_{n(t)}\mathbf{w}_f^T\mathbf{x}_{n(t)} \left\lfloor
    \frac{A(t)}{\|\mathbf{x}_{n(t)}\|^2} + 1
\right\rfloor\\
&\geq T \cdot \min_{t \in T} y_{n(t)}\mathbf{w}_f^T\mathbf{x}_{n(t)} \left\lfloor
    \frac{A(t)}{\|\mathbf{x}_{n(t)}\|^2} + 1
\right\rfloor\\
&> T \cdot \min_{t \in T} y_{n(t)}\mathbf{w}_f^T\mathbf{x}_{n(t)} \frac{A(t)}{\|\mathbf{x}_{n(t)}\|^2} = \text{const} \cdot T
\end{align*}

Combining these two inequality,
\[
\mathbf{w}_f^T\mathbf{w}_T > T \cdot \min_{t \in T} y_{n(t)}\mathbf{w}_f^T\mathbf{x}_{n(t)} \frac{A(t)}{\|\mathbf{x}_{n(t)}\|^2}, \quad
\|\mathbf{w}_T\| < \sqrt T \cdot \max_{t\in T}\|\mathbf{x}_{n(0)}\|,
\]
finally we have 
\begin{align*}
1 
&\geq \frac{\mathbf{w}_f^T}{\|\mathbf{w}_f^T\|} \cdot \frac{\mathbf{w}_T}{\|\mathbf{w}_T\|}\\
&> \frac{1}{\|\mathbf{w}_f^T\|} \cdot \frac{
    T \cdot \min\limits_{t \in T} y_{n(t)}\mathbf{w}_f^T\mathbf{x}_{n(t)} \dfrac{A(t)}{\|\mathbf{x}_{n(t)}\|^2}
}{
    \sqrt{T} \cdot \max\limits_{t\in T}\|\mathbf{x}_{n(0)}\|
}\\
&> \frac{1}{\|\mathbf{w}_f^T\|} \cdot \frac{
    \min\limits_{t \in T} y_{n(t)}\mathbf{w}_f^T\mathbf{x}_{n(t)} \dfrac{A(t)}{\|\mathbf{x}_{n(t)}\|^2}
}{
    \max\limits_{t\in T}\|\mathbf{x}_{n(t)}\|
} \cdot \sqrt{T}
\end{align*}
thus
\begin{align*}
T
&< \left[\frac{
	\|\mathbf{w}_f^T\| \cdot \max\limits_{t \in T} \|\mathbf{x}_{n(t)}\|
}{
	\min\limits_{t \in T} 
		\dfrac{y_{n(t)}\mathbf{w}_f^T\mathbf{x}_{n(t)}}{\|\mathbf{x}_{n(t)}\|} 
	\cdot 
		\dfrac{y_{n(t)}\mathbf{w}_{n(t)}\mathbf{x}_{n(t)}}{\|\mathbf{x}_{n(t)}\|}
}\right]^2\\
&\leq \left[\frac{
	\|\mathbf{w}_f^T\| \cdot \max\limits_{t \in \mathcal{D}} \|\mathbf{x}_{n(t)}\|
}{
	\min\limits_{t \in \mathcal{D}} 
		\dfrac{y_{n(t)}\mathbf{w}_f^T\mathbf{x}_{n(t)}}{\|\mathbf{x}_{n(t)}\|} 
	\cdot 
		\dfrac{y_{n(t)}\mathbf{w}_{n(t)}\mathbf{x}_{n(t)}}{\|\mathbf{x}_{n(t)}\|}
}\right]^2
\end{align*}

Therefore we have proven that number of updates $T$ has a upper bound for a given data set $\mathcal{D}$. 
\end{document}
